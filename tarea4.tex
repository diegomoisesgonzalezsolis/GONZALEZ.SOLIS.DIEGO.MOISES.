\documentclass[12pt,a4paper]{article}
\usepackage[utf8]{inputenc}
\usepackage[spanish]{babel}
\usepackage{amsmath}
\usepackage{amsfonts}
\usepackage{amssymb}
\usepackage{graphicx}
\usepackage[left=2cm,right=2cm,top=2cm,bottom=2cm]{geometry}
\author{Gonzalez Solis Diego Moises}
\title{Condiciones de singularidad de manipuladores seriales}

\begin{document}

\maketitle
\includegraphics[scale=2]{upzroja.jpg} 
\newpage
\section{introduccion}
Un manipulador serial se encuentra en una configuración singular cuando su órgano terminal es incapaz de moverse con una
velocidad arbitraria dentro del espacio de trabajo. Esta situación es en general indeseable y debe evitarse o eliminarse. La primera
opción es fácil de realizar aunque conlleva una reducción del espacio de trabajo. La mayoría de los manipuladores deben ser
capaces de operar en configuraciones singulares. Si la singularidad es inevitable, es necesario implementar un procedimiento que
permita eliminar la singularidad en que se encuentra el manipulador serial para tratar de recuperar su completa movilidad. En este
trabajo se muestra cómo el producto de Lie, una operación fundamental del álgebra de Lie que es isomórfica con el álgebra de
tornillos, permite identificar los tornillos asociados a los pares cinemáticos del manipulador, responsables de provocar la singularidad. A diferencia de otros procedimientos reportados en la literatura, el aquí empleado es aplicable a manipuladores que pierden más de un grado de libertad

\includegraphics[scale=1]{2.jpg} 

Un manipulador serial redundante posee más
grados de libertad que los estrictamente necesarios para colocar su órgano terminal dentro
del espacio de trabajo. Gracias a los citados
grados de libertad excedentes que conforman la
conocida redundancia del manipulador, éste
posee una alta capacidad de ser manipulable y
en consecuencia es capaz de ejecutar trayectorias más sofisticadas que las realizables con un
manipulador no redundante.


Una singularidad ocurre cuando el órgano terminal de un manipulador serial es incapaz de
moverse con una velocidad arbitraria dentro del
espacio de trabajo. Esta situación conduce a una
solución físicamente irrealizable del análisis de
velocidad inverso.




Baker y Wampler II (1988) desarrollaron un procedimiento basado en conceptos de topología
matemática para analizar singularidades enmanipuladores seriales redundantes que se componen exclusivamente de pares prismáticos o de
“revoluta”. El mismo Wampler II (1988) aplicó
dichos conceptos en la determinación de las
trayectorias factibles que permitan al manipulador abandonar la singularidad en que se
encuentra. Ramdane-Cherif et al. (1996) resuelven el análisis de velocidad inverso mediante
redes neuronales y de esta manera, analizan la
redundancia del manipulador.
Sin duda, la teoría de tornillos infinitesimales es
una alternativa confiable que permite de una
forma sistemática y exacta explicar la naturaleza
física de las singularidades.
Lipkin y Duffy (1985) mostraron que una singularidad está presente cuando la “cilindroide”,
generada entre dos tornillos coaxiales, degenera
en una línea.
Cuando el determinante de una matriz es cero,
condición necesaria para la existencia de la singularidad en manipuladores seriales no redundantes, existe dependencia lineal entre las
columnas que la componen. Esta situación permite ubicar como una primera aproximación
los posibles elementos que provocan la singularidad. Chevallier (1995) desarrolló un algoritmo basado en el álgebra de Lie para probar la
dependencia lineal en conjuntos de tornillos.
Podhorodeski et al. (1993) investigó la pérdida
de movilidad en manipuladores redundantes
agrupando los tornillos asociados a los pares
cinemáticos del manipulador linealmente
dependientes.
Es interesante mencionar que, si bien es cierto,
existe una cantidad bastante respetable de contribuciones que tratan el problema de las singularidades en temas como la identificación y la
caracterización de singularidades, sorprendentemente, el tema de escape de singularidades ha
llamado poco la atención.
El primer estudio formal que trata del tema del
escape de singularidades se debe a Hunt (1986),
quien recurrió a la matriz de cofactores de la
matriz Jacobiana y a la teoría de sistemas de tornillos para detectar la singularidad y al tornillo responsable de causarla. Cuando un manipulador serial no redundante pierde un grado de libertad, el rango de la correspondiente matriz
Jacobiana es 5, por lo tanto se tienen cinco
tornillos linealmente independientes con un
tornillo recíproco común. Hunt (1986) determinó
que en una configuración singular, el tornillo que
es linealmente dependiente, no interviene en la
determinación del tornillo que es recíproco. Un
resultado similar fue reportado previamente por
Sugimoto et al. (1982).
Parikian (1996) empleó los determinantes de la
matriz Jacobiana y la matriz Gramiana para
determinar la proximidad de singularidades en
manipuladores seriales no redundantes. La combinación de los gradientes de esas matrices
provee información para liberar al manipulador
de la singularidad en que se encuentra. En ese
mismo año, Karger (1996) utilizó el álgebra de
Lie para describir las posibles configuraciones
singulares en manipuladores seriales no redundantes.
En un trabajo previo Rico et al. (1995) introdujeron un método para identificar a los tornillos que
provocan dicha singularidad, dado un manipulador
serial no redundante en una configuración singular.
A diferencia de los procedimientos anteriormente
mencionados, en particular el de Hunt (1986). El
método propuesto por Rico et al. (1995) es aplicable a manipuladores no redundantes que pierden
más de un grado de libertad, tarea que se considera
además de compleja, extremadamente laboriosa.
En este trabajo se extiende el método introducido por Rico et al. (1995), aplicable en manipuladores seriales no redundantes a los manipuladores seriales redundantes.





 EJEMPLO 3, MANIPULADOR DE
SIETE GRADOS DE LIBERTAD

El manipulador serial redundante que se muestra
en la figura 3 es una adaptación de una cadena
cinemática cerrada propuesta por Sugimoto et al.
(1982). El mecanismo original fue propuesto
para su caracterización y los resultados obtenidos
demuestran que se trata de una estructura.
Con el objeto de ejemplificar la metodología que
se aplica en el presente trabajo, el mecanismo
original se “rompió” para producir una cadena
serial abierta y de esta manera, analizar las configuraciones singulares.

\includegraphics[scale=.5]{3.png}












\end{document}